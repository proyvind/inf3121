\documentclass[UKenglish]{article}  %% ... or USenglish or norsk
\usepackage[utf8]{inputenc}
\usepackage[T1]{url}

\usepackage[a4paper]{geometry}
\usepackage{listings}

\setlength{\parskip}{1em}
\setlength{\parindent}{0em}
\usepackage{caption}
\usepackage{subcaption}

\urlstyle{sf}
\usepackage{babel}
\usepackage{ifikompendiumforside}
\usepackage{hyperref}
\usepackage{tikz}
\usepackage{array}
\usepackage[font={small,it,sf}]{caption}
\usepackage{longtable}
\usepackage{wrapfig}
\usetikzlibrary{arrows, shadows}

\title{INF4121 Project Assignment 1}
\subtitle{Interpret the metrics offered by the static analyzer}
\author{Per Øyvind Karlsen, Melat Fisseha Tekelmichael}

\begin{document}
\ififorside

\section{Introducion}

\subsection{Objective}
The objective of the assignment is to discover what the source code does and
test it manually. This source code then needs to be analyzed using a static
analyzer tool called “Source Monitor”. The tool will help us obtain some
code metrics and interpret them based on the project under analysis to see its
weak parts of the code. The weak points then need to be improved and measure
it again to see the improved metrics.
The main aim will then be to compare the two interpretations of these metrics. 

\subsection{Project selection}
The Java implementation of Minesweeper was the project we chose to analyze for
our assignment and is available on github:

\url{https://github.com/proyvind/inf3121}

\section{Requirement 1 - Brief analysis}

\subsection{Brief description}
The project we chose consists of three files: “MineField.java”,
“Ranking.java” and a test program  “Minesweeper.java”.
Each file is constructed with having only one class.
The project is a game called Minesweeper where a player is initially presented
with a grid of undifferentiated squares. Some randomly selected squares,
unknown to the player, are designated to contain mines.
Typically, the size of the grid and the number of mines are set in advance by
the user, either by entering the numbers or selecting from defined skill
levels, depending on the implementation. The number of mines, is equivalent to
1/3 the number of squares, or less.

\url{https://en.wikipedia.org/wiki/Minesweeper_(video_game)}

\subsection{Analysis of testable parts}
In order to test its functionality we have used black-box testing. Here we test
the input and see the output we get. We chose the \textbf{Boundary value analysis}
and \textbf{Equivalence class partitioning} techniques.
We have also used two factors that will help us analyse the testable parts.
\textbf{Observability}: where this shows us the degree of possibility to observe the
test results.
\textbf{Understandability}: to check if the component is self-explaining. 
\textbf{Usability}: 

As non- functional testing tests the quality characteristics of the component,
it would not make sense not to write a non-functional tests. 
Though it covers the aspects of the product that may not relate to a specific
function, it still tests the aspects like reliability, usability, portability
and many more. As to the program that i have selected to this assignment, I
would say to write a non-functional test. To ground the argument let us assume
that a customer needs to include this game in his/ her casino house.
The owner that’s going to buy the product needs to know  if the these type
of tests has been performed.

\subsection{Non-functional tests}

\subsubsection{Performance, load and stress}

\subsubsection{Reliability}

\subsubsection{Usability}

\subsubsection{Efficiency}

\subsubsection{Maintainability}

\subsubsection{Portability}

\subsection{Test cases}

\end{document}
